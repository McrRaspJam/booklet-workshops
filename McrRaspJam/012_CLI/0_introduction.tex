\setcounter{section}{-1}
\section{Introduction}
	
	%Workshop Description
	This workshop takes a look at the basics of the Linux Command Line Interface (\textbf{CLI}), primarily to be able to configure and maintain our Pi's software.
	
	On a Linux computer the desktop only scratches the surface of the programs and functionality available. We'll only be able to take a look at the very basics today, but for each new trick you learn in the CLI, you'll find a new way to save a little time when using your computer.

	%Difficulty
	This is an intermediate Workshop. Once you're familiar with using your Pi on the desktop, you should be ready to attempt this workshop.
		
	\subsection*{How to use these Booklets}
	
	%Code Listings
	When you need to type a command in the command line interface, they will be listed like the following example. Copy everything \textit{after} the dollar sign. Lines without dollar signs are example outputs, and do not need to be copied.
			
\begin{lstlisting}[style=Terminal]
$ echo Hello, World!
Hello, World!
\end{lstlisting}
	Occasionally, something will be explained in greater detail in asides, like the one below. You can read these as you wish.
	
\begin{aside}
	The command line interface isn't really a built-in part of the Linux operating system. It's a type of program called a `shell', that runs on top of the operating system.
	
	Raspbian uses `bash' (\textbf{B}ourne-\textbf{A}gain \textbf{SH}ell), which is the most typical shell in Linux distributions, but you could choose to install and alternative shell.
\end{aside}
		
	%Aknowledgements
	These booklets were created using {\fontfamily{rfdefault}\selectfont \LaTeX}, an advanced typesetting system used for several sorts of books, academic reports and letters.
		
	%License spiel
	To allow modification and redistribution of these booklets, they are distributed under the \hbox{CC BY-SA 4.0} License. LaTeX source documents are available at \url{http://github.com/McrRaspJam/booklet-workshops}
	
	All of our workshop resources are available to download from a Google Drive at
\url{http://bit.ly/mcrraspjam}.
There, you can find a PDF copy of this booklet, as well as template and completed program files for each workshop.
	
	\subsection*{What you'll need}
		
		Just a working Raspberry Pi running Raspbian
		
	\subsection*{Questions?}
		If you get stuck, find errors or have feedback about these booklets, email:
		\url{jam@jackjkelly.com}\label{email}