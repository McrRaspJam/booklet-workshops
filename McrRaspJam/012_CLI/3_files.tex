\section{Files}
	
	Once you've picked a directory, you can use the \textbf{ls} command, which is short for the word list.
	
	\begin{lstlisting}[style=Terminal]
 $ cd
~$ ls
Desktop Downloads Pictures python_games
	\end{lstlisting}

	Once we've found the file we're looking for, it's straightforward to edit it. For example, a text file could be opened using
	
	\begin{lstlisting}[style=Terminal]
~$ nano textfile.txt
	\end{lstlisting}	

	If a file doesn't exist, most editor programs will create a new file with that name. Save this new file by pressing \textbf{Ctrl+O}, then quitting with \textbf{Ctrl+X}. Run ls again, and your file should now be present.

	\subsection{Manipulating Files}
	
		Let's blitz through a few file operation commands.
		
		To copy a file:
		\begin{lstlisting}[style=Terminal]
~$ cp textfile.txt anothertextfile.txt
		\end{lstlisting}

		To move a file:
		\begin{lstlisting}[style=Terminal]
~$ cp textfile.txt originaltextfile.txt
		\end{lstlisting}

		You'll note that both of these commands use the same parameters, the original file followed by the resulting file.

		To remove a file:	
		\begin{lstlisting}[style=Terminal]
~$ rm anothertextfile.txt
		\end{lstlisting}

		To make a directory:		
		\begin{lstlisting}[style=Terminal]
~$ mkdir textfile
		\end{lstlisting}

		We should be left with originaltextfile.txt and a directory. Try \textit{moving} the text file to within this new directory.
		
		you can remove an empty directory with rmdir, but because ours has a file in it, we will delete it using the following command.
		\begin{lstlisting}[style=Terminal]
~$ rm -r textfile
		\end{lstlisting}

		-r is flag that tells rm to run recursively. There are lots of flags for most commands, for example try \textbf{ls -l}. you can view the flags of a program by viewing it's manual using the \textbf{man} command.
		\begin{lstlisting}[style=Terminal]
~$ man ls
		\end{lstlisting}
		
		