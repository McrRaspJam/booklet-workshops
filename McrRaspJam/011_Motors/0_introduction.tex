\setcounter{section}{-1}
\section{Introduction}
	
	
	%Workshop Description	
	The workshop shows how to control a motor with the Raspberry Pi, using a breadboard circuit and a standard motor controller chip. The motor is then programmed with the Python GPIO library.

	%Difficulty
	This is and introductory workshop and is suitable for anybody to attempt.
		
	%Aknowledgements
	These booklets were created using {\fontfamily{rfdefault}\selectfont \LaTeX}, an advanced typesetting system used for several sorts of books, academic reports and letters. The schematic diagrams were created using \textit{Fritzing}, a free hardware prototyping and PCB design software.
		
	%License spiel
	To allow modification and redistribution of these booklets, they are distributed under the \hbox{CC BY-SA 4.0} License. LaTeX source documents are available at \url{http://github.com/McrRaspJam/booklet-workshops}
	
	
	\subsection*{What you'll need}
		
			
		The hardware requirements for this workshop are listed below.
	
		\begin{itemize}[nosep]		
			\item A solderless breadboard, preferably one with separate bus strips. (those labelled + \& -)
			\item Male-to-female \& male-to-male jumper cables.
			\item A 6V DC motor.
			\item A L293D or L293DNE motor controller.
			\item (Optional) a 4$\times$AA Battery Pack				
		\end{itemize}

		For the LED exercise in \autoref{sec:LED}, you will also require:
			
		\begin{itemize}[nosep]	
			\item An LED
			\item A $\sim 330 \Omega$ Resistor
		\end{itemize}
				
	
	%File Downloads	
	%All of our workshop resources are available to download from a Google Drive at
\url{http://bit.ly/mcrraspjam}.
There, you can find a PDF copy of this booklet, as well as template and completed program files for each workshop.	
			
	%Code Listings
	\subsection*{Code listings \& asides}
	
	When you need to make changes to your code, they'll be presented in boxes like the following:

	\lstinputlisting[style=Python]{McrRaspJam/011_Motors/0_introduction/helloworld.py}
	
	You might not need to copy everything, so check the line numbers to make sure you're not copying something twice.
	%Terminal commands are listed as such, copy everything \textit{after} the dollar sign. Lines without dollar signs are example outputs, and do not need to be copied.
			
\begin{lstlisting}
$ cd code
$ python3 helloworld.py 
Hello, World!
Hello, World!
\end{lstlisting}
	%Occasionally, a concept will be explained in greater detail in \textit{asides}, like the one below. You can read these as you wish, but they're not required to complete the workshop.
	
\begin{aside}[Resistors]
	Resistors are most commonly used to limit the amount of current flowing through part of an electrical circuit.
	
	For example we use resistors in series with LEDs, as otherwise they could draw so much power that they destroy themselves.
	
	Buttons have almost no internal resistance, so we use high value ($\sim 10  k\Omega$) resistors to prevent current flowing straight from the power supply to ground; if we didn't, the entire CPU could be short circuited, and the Pi would lose power!
\end{aside}
		
	\subsection*{Questions?}
		If you get stuck, find errors or have feedback about these booklets, email:
		\url{jam@jackjkelly.com}\label{email}