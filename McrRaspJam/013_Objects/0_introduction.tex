\setcounter{section}{-1}
\section{Introduction}
	
	Object-oriented programming is a different way to represent complex concepts than the traditional programming methods. Today we'll take a look at what objects are using the Java programming language.
	
	Then, we'll use objects to create characters for a computerised tabletop role-playing game.

	%Difficulty
	This is an intermediate workshop, that builds upon a familiar knowledge of a text-based programming language like Python.
		
	\subsection*{How to use these Booklets}

	The aim of these booklets is to help you attempt these workshops at home, and to explain concepts in more detail than at the workshop. You don't need to refer to use these booklets during the workshop, but you can if you'd like to.
	
	%Code Listings
		When you need to make changes to your code, they'll be presented in boxes like the following:

	\lstinputlisting[style=Java, firstline=1, firstnumber=1, lastline=4]{McrRaspJam/013_Objects/0_introduction/HelloWorld.java}
	
	You might not need to copy everything, so check the line numbers to make sure you're not copying something twice.
	
	When you need to type a command in the command line interface, they will be listed like the following example. Copy everything \textit{after} the dollar sign. Lines without dollar signs are example outputs, and do not need to be copied.
			
\begin{lstlisting}[style=Terminal]
$ java HelloWorld
Hello, World!
\end{lstlisting}
	Occasionally, something will be explained in greater detail in asides, like the one below. You can read these as you wish, but they're not required to complete the workshop.
	
\begin{aside}[Java byte code]
	Programming languages are traditionally either compiled or interpreted. Compiled languages like C are turned into architecture-specific machine code. Interpreted languages like Python is loaded by their interpreter and ran line-by-line, which is usually a lot slower.
	
	In an attempt to get the flexibility of interpreted language with most of the performance of compiled languages, Java code is compiled into an efficient `byte code', which is not specific to any architecture, but is instead run through the Java virtual machine.
\end{aside}
		
	%Aknowledgements
	These booklets were created using {\fontfamily{rfdefault}\selectfont \LaTeX}, an advanced typesetting system used for several sorts of books, academic reports and letters.
		
	%License spiel
	To allow modification and redistribution of these booklets, they are distributed under the \hbox{CC BY-SA 4.0} License. LaTeX source documents are available at \mbox{\href{http://github.com/McrRaspJam/booklet-workshops}{github.com/McrRaspJam/booklet-workshops}}
	
	%All of our workshop resources are available to download from a Google Drive at
\url{http://bit.ly/mcrraspjam}.
There, you can find a PDF copy of this booklet, as well as template and completed program files for each workshop.
	
	\subsection*{What you'll need}
		
		We'll be using Java for this workshop, which is installed by default in Raspbian.
		
	\subsection*{Questions?}
		If you get stuck, find errors or have feedback about these booklets, please email me at:
		\mbox{\url{jam@jackjkelly.com}\label{email}}