\section{Java quick start}

	In your preferred editor, create a new file and copy the following code. Note that Java is case sensitive.
	
	\lstinputlisting[style=Java, breaklines=false]{McrRaspJam/013_Objects/1_java/HelloWorld.java}
	
	Save this file as \textbf{HelloWorld.java}, then open a terminal to the same folder, and type the commands listed below. You should get the following output.
	
	\begin{lstlisting}[style=Terminal]
$ javac HelloWorld.java
$ java HelloWorld
Hello, World!
	\end{lstlisting}
	
	Now, compare that to the equivalent Python program:
	
	\begin{lstlisting}[style=Python]
print("Hello, World")
	\end{lstlisting}
	
	Coming from most over languages, Java may seem quite complicated. You'll find out that in Java, we have to use a lot of terms that would be optional or excluded in other languages. Java is a language where seemingly nothing is implied.
	
	\begin{aside}[println()]
		println() is the print function for Java. In the Java language we have to state which library or class this function comes from, which in this case is from the \texttt{System} class. \texttt{out} is the standard output stream, text sent to this stream is printed in the terminal.
	\end{aside}
	
	We'll explain what some of the other lines are soon. For now, each program we write will have these same lines, so we don't need to worry about them just yet.
	
	\subsection{Variables}
	
	In Java, variables must be defined with a type. Common types might include:
	
	\begin{itemize}[nosep]
		\item \texttt{int} -- An integer (whole number).
		\item \texttt{double} -- A double-precision floating point number i.e. a number with a decimal point.
		\item \texttt{boolean} -- A boolean value i.e. true or false.
		\item \texttt{String} -- A String i.e text. Strings are a collection of type \texttt{char}.
	\end{itemize}

	An example of variables being defined, then modified is shown below.
	
	\begin{lstlisting}[style=Java, numbers=none]
int a;
a = 5;
int b = 3;
a = a + b;
	\end{lstlisting}
	
	What would the values stored in \texttt{a} and \texttt{b} be after this code was run?	
	
	\subsection{Loops and Conditional Statements}
	
		If you've done any programming before, you'll know that we frequently use loops, or \textbf{iterative statements} to perform repetitive tasks. The \texttt{for} statement is a loop that will operate a set number of times based on what parameters we provide it.
		
		\begin{lstlisting}[style=Java, firstnumber=5]
for(int i=0; i<10; i++)
{
	// Code to repeat
}
else
		\end{lstlisting}
	
		The above \texttt{for} loop would loop it's contents 10 times, starting with \texttt{i} at 0, and stopping when \texttt{i} reaches 10. \texttt{i++} is a shorthand notation for \texttt{i = i+1}, so each time the loop ends \texttt{i} is increased by 1.
		
		Let's modify our hello world program to print multiple times using this loop. Make the changes in the listing below, then compile (\texttt{javac}) and run (\texttt{java}) your program as before, and you should now see ``Hello, World!'' printed 10 times.
		
		\lstinputlisting[style=Java]{McrRaspJam/013_Objects/1_java/HelloWorldLoop.java}
		
		A \textbf{conditional statements} like \texttt{if else} are used to selectively run code based on a condition. Let's use an \texttt{if else} statement to change our program a little. We'll change the \texttt{println()} to print out \texttt{i}, the current count number. Then, when we reach a specific number, we'll print out a bit of text after the number.
		
		\lstinputlisting[style=Java, firstline=5, firstnumber=5, lastline=11]{McrRaspJam/013_Objects/1_java/HelloWorldIf.java}
		
		Now compile and run the program again, you should see something like
		
		\begin{lstlisting}[style=Terminal, numbers=none]
...
4
5
My favourite number!
6
...
		\end{lstlisting}
		
		\textbf{Optional:} A standard basic programming test is FizzBuzz. Use loops and conditions to recreate the game.
		
		\iffalse		
		If you want to try to make something on your own with these statements, a common programming exercise with loops and conditions is FizzBuzz. You'll need the modulus operator \texttt{\%} for your conditions.
		
		\begin{aside}[FizzBuzz]
			FizzBuzz is a number game that is often used as a basic example of a programming task. The objective is to count up from 1 to a certain number, but for each number that is divisible by 3 say `Fizz', for each number divisible by 5 say `Buzz', and for numbers that are divisible by both numbers, say `FizzBuzz'.
			
			The answer for FizzBuzz starts 1, 2, Fizz, 4, Buzz, Fizz, 7, 8, Fizz, Buzz, 11, Fizz, 13, 14, FizzBuzz, 16...
		\end{aside}
		\fi
	
	\subsection{Methods}
	
		\begin{aside}[Sequential Execution]
			A traditional computer program runs one line at a time from top to bottom. In a language like Java, the \textit{main method} is the thing that runs from start to end, and the program ends when the main method is complete. We use method calls to `extend' the main method by jumping to additional code.
		\end{aside}
	
		Java also has \textbf{methods}, often called \textbf{functions} or \textbf{procedures} in other languages, used in much the same way.
		
		Let's write a method that prints out a square number. In our current program, we'll define a new method like the following listing. We also need to make a call to the function from \texttt{main()}.  When you run your program again, it should now print `Hello, World!' 10 times.
		
		\lstinputlisting[style=Java, firstline=3, firstnumber=3,  lastline=14]{McrRaspJam/013_Objects/1_java/FunctionA.java}
		
		\textbf{Parameters} are passed in the same way as many other languages, within the parentheses with a type deceleration. Functions can also \textbf{return} a value, but the method deceleration must be changed to say what type it returns.
		
		Currently the function is \texttt{void}, meaning it returns nothing. To return a number, we need to change \texttt{void} to \texttt{int}. Let's make the function return the square of the parameter input, then \texttt{main()} will print that number.
		
		\lstinputlisting[style=Java, breaklines=false, firstline=3, firstnumber=3, lastline=15]{McrRaspJam/013_Objects/1_java/FunctionB.java}
	
	