Occasionally, something will be explained in greater detail in asides, like the one below. You can read these as you wish, but they're not required to complete the workshop.
	
\begin{aside}[Java byte code]
	Programming languages are traditionally either compiled or interpreted. Compiled languages like C are turned into architecture-specific machine code. Interpreted languages like Python is loaded by their interpreter and ran line-by-line, which is usually a lot slower.
	
	In an attempt to get the flexibility of interpreted language with most of the performance of compiled languages, Java code is compiled into an efficient `byte code', which is not specific to any architecture, but is instead run through the Java virtual machine.
\end{aside}