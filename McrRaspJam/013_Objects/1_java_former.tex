\section{Java quick start}

	In your preferred editor, create a new file and copy the following code.
	
	\lstinputlisting[style=Java]{McrRaspJam/013_Objects/1_java/HelloWorld.java}
	
	Save this file as \textbf{HelloWorld.java}, then open a terminal to the same folder, and type the commands listed below. You should get the following output.
	
	\begin{lstlisting}[style=Terminal]
$ javac HelloWorld.java
$ java HelloWorld
Hello, World!
	\end{lstlisting}
	
	Now, compare that to the equivalent Python program:
	
	\begin{lstlisting}[style=Python]
print("Hello, World")
	\end{lstlisting}
	
	Coming from most over languages, Java may seem quite complicated. You'll find out that in Java, we have to use a lot of terms that would be optional or excluded in other languages. Java is a language where seemingly nothing is implied.
	
	\begin{aside}[println()]
		println() is the print function for Java. In the Java language we have to state which library or class this function comes from, which in this case is from the \texttt{System} class. \texttt{out} is the standard output stream, text sent to this stream is printed in the terminal.
	\end{aside}
	
	We'll explain what some of the other lines are soon. For now, each program we write will have these same lines, so we don't need to worry about them just yet.
	
	\subsection{Loops and Conditional Statements}
	
		If you've done any programming before, you'll know that we frequently use loops, or \textbf{iterative statements} to perform repetitive tasks. The \texttt{for} statement is a loop that will operate a set number of times based on what parameters we provide it.
		The format for the \texttt{for} loop is \mbox{\texttt{for(start; end; increment)}}.
		
		\begin{lstlisting}[style=Java, firstnumber=5]
for(int i=0; i<10; i++)
{
	// Code to repeat
}
else
		\end{lstlisting}
	
		The above \texttt{for} loop would loop it's contents 10 times, starting with \texttt{i} at 0, and stopping when \texttt{i} reaches 10. \texttt{i++} is a shorthand notation for \texttt{i = i + 1}, so each time the loop ends \texttt{i} is increased by 1.
		
		Let's modify our hello world program to print multiple times using this loop. Make the changes in the listing below, then compile (\texttt{javac}) and run (\texttt{java}) your program as before, and you should now see ``Hello, World!'' printed 10 times.
		
		\lstinputlisting[style=Java, breaklines=true]{McrRaspJam/013_Objects/1_java/HelloWorldLoop.java}
		
		A \textbf{conditional statement} are those that only executes if a required condition is met. Most languages have an \texttt{if} statement, and Java's works the same way as most other languages, with the format \texttt{if(condition)}. We also have \texttt{else}, if we want to perform two different things based on the result of our condition.
		
		Let's change our program a little. We'll change the \texttt{println} to print out \texttt{i}, the current count number. Then, when we reach a specific number, we'll print out a bit of text after the number.
		
		\lstinputlisting[style=Java, breaklines=true, firstline=5, firstnumber=5, lastline=11]{McrRaspJam/013_Objects/1_java/HelloWorldIf.java}
		
		Now compile and run the program again, you should see something like
		
		\begin{lstlisting}[style=Terminal]
...
4
5
My favourite number!
6
...
		\end{lstlisting}
		
		If you want to try to make something on your own with these statements, a common programming exercise with loops and conditions is FizzBuzz. You'll need the modulus operator \texttt{\%} for your conditions.
		
		\begin{aside}[FizzBuzz]
			FizzBuzz is a number game that is often used as a basic example of a programming task. The objective is to count up from 1 to a certain number, but for each number that is divisible by 3 say `Fizz', for each number divisible by 5 say `Buzz', and for numbers that are divisible by both numbers, say `FizzBuzz'.
			
			The answer for FizzBuzz starts 1, 2, Fizz, 4, Buzz, Fizz, 7, 8, Fizz, Buzz, 11, Fizz, 13, 14, FizzBuzz, 16...
		\end{aside}
	
	\subsection{Methods}
	
	\textbf{Methods}, often called \textbf{functions} in other languages, are used to separate parts of our code. This is useful for several reasons -- like readability and repetition -- but is primarily used to seperate our code into individual tasks (hence function).
	
	Let's write a method that prints out a square number. In our current program, we'll define a new method like so:
	
	\lstinputlisting[style=Java, breaklines=true, lastline=9]{McrRaspJam/013_Objects/1_java/FunctionA.java}
	
	You can see that we already had a method, called \texttt{main()}. This is the \textbf{main method}, and is the first thing that is run when Java executes our program.
	
	We've added a new method called \texttt{printHelloWorld()}, which will print `Hello, World!' when run. Try compiling and running now. Currently, the program behaves as before, and does not print `Hello, World'. This is because we need to \textbf{call} the method to run it.
	
	\begin{aside}[Sequential Execution]
		A traditional computer program runs one line at a time from top to bottom. In a language like Java, the \textit{main method} is the thing that runs from start to end, and the program ends when the main method is complete. We use method calls to `extend' the main method.
	\end{aside}

	Modify your main method so that it calls the new method within the \texttt{for} loop as show below. When you run your program again, it should now print `Hello, World!' 10 times.
	
	\lstinputlisting[style=Java, breaklines=true, firstline=8, firstnumber=8, lastline=14]{McrRaspJam/013_Objects/1_java/FunctionA.java}
	
	We can provide \textbf{parameters} to a method, which allow us to pass values to the method that can allow them to perform more useful tasks. Let's change our method to calculate and print a squared number. Make the following changes, note that we've changed the method name, both where it is defined \textit{and} where it is called, and we've set \texttt{i} as the parameter of our method call.
	
	\lstinputlisting[style=Java, breaklines=true, firstline=3, firstnumber=3, lastline=15]{McrRaspJam/013_Objects/1_java/FunctionB.java}
	
	
	
	