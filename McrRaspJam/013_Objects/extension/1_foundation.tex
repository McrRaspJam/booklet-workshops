\section{Setup}

	Firstly, connect the Pi to the Internet using an Ethernet cable. If you want to use Wi-Fi instead, you will have to read on how to configure your wireless receiver using the command line after your Pi has finished booting.
	
	Boot your Pi with the Raspbian Lite SD card. The credentials to log in are as follows:
	\\
	\\\textbf{username:}\hspace{2cm}pi
	\\\textbf{password:}\hspace{2cm}raspberry
	
	\vspace{10pt}
	Once you're logged in, it'll be a good idea to check you're connected to the internet. Type:

		\begin{lstlisting}[style=Terminal]
$ ping www.google.com
PING www.google.com (216.58.213.100) ...
		\end{lstlisting}

	Ping will repeatedly try to contact a server (in this case, the Google webpage), and the server will notice this and send a response. Press \textbf{Ctrl + C} to stop pinging.
	
		\subsection*{apt-get}
		
		From the previous command line workshop, you may recall we use a program called `apt-get' to install software from within the command line.
		
		Apt-get stores a list of available software, which we can then choose to install from. To get an up-to-date list of available software, you'll \textbf{update} this list using
	
		\begin{lstlisting}[style=Terminal]
$ sudo apt-get update
		\end{lstlisting}

		After a few moments, the \$ prompt will reappear, telling us the command has completed.
		
		You can take the opportunity now to update the software on the Pi, you do this using an \textbf{upgrade} command.
	
		\begin{lstlisting}[style=Terminal]
$ sudo apt-get upgrade
		\end{lstlisting}	

		On Raspbian, this command would take quite a while. Luckily on Raspbian Lite, there's not a lot of software to be upgraded!
		
	That's it, the foundation has been built! The Pi is up to date and ready to be used now. Well, obviously without the GUI, we have built the house but there's no furniture inside. 