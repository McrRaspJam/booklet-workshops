\section{Substitution Ciphers}

	\ifprint\else	
		Ciphers have existed in some form since the classical era. One of the earliest examples of ciphers being used to keep information are clay tablets from Mesopotamia---dating back to 1500 BCE---which were used to keep trade secrets like the recipe for pottery glaze.
	
		\begin{aside}[Origins of Ciphers]
			From \href{http://en.wikipedia.org/wiki/History_of_cryptography}{Wikipedia}:
			
			\textit{``The earliest known use of cryptography is found in non-standard hieroglyphs carved into the wall of a tomb from the Old Kingdom of Egypt circa 1900 BCE.}
				
			\textit{``These are not thought to be serious attempts at secret communications, however, but rather to have been attempts at mystery, intrigue, or even amusement for literate onlookers.''}
		\end{aside}
	\fi
	
\subsection{Atbash}

	Atbash was a simple substitution cipher that emerged around 500 BCE, and was originally used to encode the Hebrew alphabet.
	\ifprint\else (the name is a shortening of the letters Aleph-Tav-Beth-Shin) \fi
	
	In a substitution cipher, the characters of the plain-text are \textit{substituted} with another set of characters---usually the same alphabet in a different order. For Atbash, this is very simple, the alphabet is reversed.
	
	\begin{figure*}
		\begin{tabular}{r|cccccccccccccccccccccccccc}
			\textbf{plaintext} & a & b & c & d & e & f & g & h & i & j & k & l & m & n & o & p & q & r & s & t & u & v & w & x & y & z \\ 
			\hline
			\textbf{ciphertext} & z & y & x & w & v & u & t & s & r & q & p & o & n & m & l & k & j & i & h & g & f & e & d & c & b & a \\ 
		\end{tabular} 
		\caption{Atbash substitution for the Latin alphabet}
		\label{fig:atbash}
	\end{figure*}

	\subsubsection*{An Example}
	
		The tradition when practising cryptography is to use faux military commands, so we'll start with the plaintext ``\textbf{Attack at dawn}''.
		
		using the table in \autoref{fig:atbash}, we find each letter in the first row, and replace it with the letter in the following row.
		
		\begin{tabular}{ccc}
			A & $\rightarrow$ & Z \\ 
			t & $\rightarrow$ & g \\ 
			t & $\rightarrow$ & g \\ 
			a & $\rightarrow$ & z \\ 
			c & $\rightarrow$ & x \\ 
			k & $\rightarrow$ & p \\ 
		\end{tabular} 
	
		Our ciphertext for this phrase would be ``\textbf{Zggzxp zg wzdm}''.
	
		Now, using a pen and paper and \autoref{fig:atbash}, have a go at encoding the following plaintext using Atbash.
		
		``\textbf{We are discovered, flee at once}''
	
	\subsubsection*{Atbash in Python}
	
		Now we know how Atbash works, we can try writing a Python program that encodes it.
		
		When run in a terminal, the program should look like the following:
		
		\begin{lstlisting}[style=Terminal, numbers=none]
$ python3 atbash.py
Enter plaintext: attackatdawn
zggzxpzgwzdm
		\end{lstlisting}
		
		\textbf{Open} a new script file in IDLE by going to \texttt{File $\rightarrow$ New File}.
		
		The first thing our program needs to do when run is get the plaintext. We'll use an \texttt{input()} function, just like we did in \autoref{python}.
		
		\lstinputlisting[style=Python, lastline=1, title=atbash.py]{McrRaspJam/016_Ciphers/2_ciphers1/atbash1.py}
		
		We'll also set an empty \texttt{string} variable, where we'll add our ciphertext.
		
		\lstinputlisting[style=Python, firstline=2, firstnumber=2]{McrRaspJam/016_Ciphers/2_ciphers1/atbash1.py}
	
		The other thing we need is a copy of the table in \autoref*{fig:atbash}, which we'll place in lists.
		
		We could do this by manually defining the list:
		
		\lstinputlisting[style=Python, numbers=none, breaklines=true, lastline=1]{McrRaspJam/016_Ciphers/2_ciphers1/alphabetlist.py}
		
		but let's save some time and take advantage of one of Python's libraries to automate the generation of this list:
		
		\lstinputlisting[style=Python, lastline=6]{McrRaspJam/016_Ciphers/2_ciphers1/atbash2.py}
		
		Then we'll create the substitute alphabet by copying the list in reverse:
		
		\lstinputlisting[style=Python, firstline=7, firstnumber=7, lastline=7]{McrRaspJam/016_Ciphers/2_ciphers1/atbash2.py}
		
		\ifprint\else You can check these lists are correct by running the program, then typing \texttt{alphabet} and \texttt{subalphabet} into the shell window prompt, which will print their contents. \fi
	
		We have everything we need, we can now perform the encryption. Just like on pen and paper, we convert each letter one-by-one, so we'll use the loop we learned in \autoref{python}.
	
		\lstinputlisting[style=Python, firstline=8, firstnumber=8, lastline=9]{McrRaspJam/016_Ciphers/2_ciphers1/atbash2.py}
		
		\pagebreak[3] %pagebreak hint
		
		%\textbf{Look away} from your program for the moment.
		
		At this point we have one more problem to consider. To pull a letter from the alphabet list, we need to access it using an index number. (In \texttt{alphabet} 0=a, 1=b, 2=c, etc.)
		
		We need a way to number letters based on their position in the alphabet. Fortunately, the ASCII encoding scheme already assigns a number to each letter, and because it is widely used it is very easy to get the  code for a letter in Python.
		
		\pagebreak[2] %pagebreak hint
		
		\begin{aside}[ASCII]
			ASCII (\textit{American Standard Code for Information Interchange}) was invented as a way to standardise how letters were stored on different file formats and computer systems, starting in the early 1960's.
			
			ASCII encodes a small number of characters, mostly, alphanumeric characters, punctuation and control characters (things like `new line', `end of file' or `backspace'.)
			
			These days, most text is actually encoded in the much larger \href{https://unicode.org/}{Unicode} standard.
		\end{aside}
		
		To get a number for each letter in our loop:
		
		\begin{lstlisting}[style=Python, firstnumber=9]
for letter in plaintext:
    charnumber = ord(letter)
		\end{lstlisting}
		
		However, the ASCII number for ``a'' is `97', ``b'' is `98' etc. so we'll need to subtract 97 to match the 0--25 range of our alphabet list 
		
		\lstinputlisting[style=Python, firstline=10, firstnumber=10, lastline=10]{McrRaspJam/016_Ciphers/2_ciphers1/atbash2.py}
		
		Now, we can simply take the character number, and access the substitute alphabet list using the same index.
		
		\begin{lstlisting}[style=Python, firstnumber=11]
    subalphabet[charnumber]
		\end{lstlisting}
		
		Remember that we're in a loop, so we just need to add the current cipher letter to the ciphertext.
		
		\lstinputlisting[style=Python, firstline=11, firstnumber=11, lastline=11]{McrRaspJam/016_Ciphers/2_ciphers1/atbash2.py}
		
		Once the loop has completed, our ciphertext should be complete! We simply need to add a print statement outside of the loop.
		
		\lstinputlisting[style=Python, firstline=11, firstnumber=11]{McrRaspJam/016_Ciphers/2_ciphers1/atbash2.py}
		
		Try testing the program using the two plaintexts from the pen and paper exercise and see if the output from your program matches.
		
		Bear in mind your program currently has some limitations:
		\begin{itemize}[nosep]
			\item Capital letters won't work, as they have different ASCII codes to lowercase letters.
			\item We haven't told the program what to do with spaces, so we can't use them either.
		\end{itemize}
	
		If you wanted to extend your program, fixing these would be a good place to start. You should also have a think about the following:
		
		\begin{itemize}[nosep]
			\item Our program currently \textit{encodes} Atbash ciphers. What would we have to do to write a program that \textit{decodes} the cipher?
			\item How secure do you think Atbash is?
		\end{itemize}