\setcounter{section}{-1}
\section{Introduction}

	Last month in \href{http://mcrraspjam.org.uk/workshop-15-gpio-zero/}{workshop 15}, we \textit{encoded} a message in Morse code, a method that makes it simpler to transmit written language through rudimentary means, such as a spotlight, or a pulse in an electrical wire.
	
	We also often use the word \textit{code} to refer to an encrypted \textbf{cipher}. A different type of encoding, when we encode a message in a cipher, we attempt to make the message difficult---or ideally impossible--- for an outsider to read.
	
	Today, we'll have a go at encoding, decoding and cracking some famous ciphers of the classical era using Python programs. %, as well as taking a look at more modern ciphers and some of the cryptography tools available to us as users of Linux on the Raspberry Pi.
	
	
	%Difficulty
	This is an introductory workshop, all of the Python concepts will be covered from scratch.
		
	\subsection*{How to use these booklets}

	The aim of these booklets is to help you attempt these workshops at home, and to explain concepts in more detail than at the workshop. You don't need to follow these booklets during the workshop, but you can if you'd like 0to.
	
	%Code Listings
		When you need to make changes to your code, they'll be presented in \textit{listings} like the example below. Some lines may be repeated across multiple listings, so check the line numbers to make sure you're not copying something twice.

	\lstinputlisting[style=Python, lastline=2]{McrRaspJam/015_GPIOZero/0_introduction/led.py}
	
	
	
	Occasionally, a concept will be explained in greater detail in \textit{asides}, like the one below. You can read these as you wish, but they're not required to complete the workshop.
	
\begin{aside}[Resistors]
	Resistors are most commonly used to limit the amount of current flowing through part of an electrical circuit.
	
	For example we use resistors in series with LEDs, as otherwise they could draw so much power that they destroy themselves.
	
	Buttons have almost no internal resistance, so we use high value ($\sim 10  k\Omega$) resistors to prevent current flowing straight from the power supply to ground; if we didn't, the entire CPU could be short circuited, and the Pi would lose power!
\end{aside}
	
	\ifprint\else
		\subsection*{What you'll need}
		All of the software you for this workshop is pre-installed on recent versions of Raspbian.
	\fi
	
	\subsection*{Terminology}
	
		Some of the words used to describe programming and cryptography often cross, so the following words will be used in this booklet.
		
		\begin{tabular}{rl}
			\textbf{Program} & Code written in Python \\
			\textbf{Cipher} & A cryptographic code \\
			\textbf{Plaintext} & A phrase that \textit{isn't} encrypted \\ 
			\textbf{Ciphertext} & A phrase that \textit{is} encrypted
		\end{tabular} 
	
		\ifprint\else This should avoid confusing sentences, like the fact that last month we were coding programs to encode Morse code, this month we're coding cipher codes, a different type of encoding. \fi
	
	\subsection*{Everything else}
	
		% Aknowledgements
		\ifprint\else
			These booklets were created using \textrm{\LaTeX}, an advanced typesetting system used for several sorts of books, academic reports and letters.
			
			If you'd like to have a look at using LaTeX, We recommend looking at \TeX studio, which is available on most platforms, and also in the Raspbian repository.
		\fi
		
		% License spiel
		To allow modification and redistribution of these booklets, they are distributed under the \hbox{CC BY-SA 4.0} License.
		Latex source documents are available at \url{http://github.com/McrRaspJam/booklet-workshops}
		
		If you get stuck, find errors or have feedback about these booklets, please email me at:
		\href{mailto:jam@jackjkelly.com}{\texttt{jam@jackjkelly.com}}