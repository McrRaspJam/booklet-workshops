\setcounter{section}{-1}
\section{Introduction}

	If you've learnt the basics of a programming language, but are looking for some real problems to solve, 

	%Python is a computer programming language designed for general-purpose use and can be found in use almost everywhere. Python was designed to be easy to read and write, making it the perfect language to start off with!

	%Python has been around since 1991 and grew to be popular in 2003, where it was ranked in the top ten most popular programming languages (and it's still there!). It was the programming language of the year in 2007 and 2010 and even comes pre-installed on your Pi!

	%Difficulty
	This is an introductory workshop. Once you're familiar with using your Pi on the desktop, you should be ready to attempt this workshop.

	\subsection*{How to use these booklets}

		The aim of these booklets is to help you attempt these workshops at home, and to explain concepts in more detail than at the workshop. You don't need to follow these booklets during the workshop, but you can if you'd like 0to.

		%Code Listings
		\subsection*{Code listings \& asides}
	
	When you need to make changes to your code, they'll be presented in boxes like the following:

	\lstinputlisting[style=Python]{McrRaspJam/011_Motors/0_introduction/helloworld.py}
	
	You might not need to copy everything, so check the line numbers to make sure you're not copying something twice.
		Occasionally, something will be explained in greater detail in asides, like the one below. You can read these as you wish.
	
\begin{aside}
	Red lines beginning with number signs (\#) are called \textbf{comments}.
	They are ignored by the program, so we can use them to label pieces of code.
\end{aside}

		%All of our workshop resources are available to download from a Google Drive at
\url{http://bit.ly/mcrraspjam}.
There, you can find a PDF copy of this booklet, as well as template and completed program files for each workshop.

	\subsection*{What you'll need}

		For this workshop, you'll just need a working Raspberry Pi running a Linux operating system like Raspbian with Python installed. Luckily for us, Raspbian (and many other Linux distributions) include Python right out of the box.

	\subsection*{Everything else}

		%Aknowledgements
		\ifprint\else These booklets were created using \textrm{\LaTeX}, an advanced typesetting system used for several sorts of books, academic reports and letters. \fi

		%License spiel
		To allow modification and redistribution of these booklets, they are distributed under the \hbox{CC BY-SA 4.0} License.
		Latex source documents are available at \url{http://github.com/McrRaspJam/booklet-workshops}
