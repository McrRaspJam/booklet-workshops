\setcounter{section}{-1}
\section{Introduction}

	If you've controlled electronics using the Raspberry Pi's GPIO before, you probably used a Python library called \textit{RPi.GPIO} to write your programs.
	
	GPIO Zero is an alternative GPIO library for Python, written by Raspberry Pi's Ben Nuttall, with the help of Dave Jones. It is designed to be easier to use than RPi.GPIO, whilst having the same capabilities.

	Today, we'll learn how to create simple breadboard circuits using the Raspberry Pi's GPIO, then we'll write Python programs that control those circuits using the GPIO Zero library.

	%Difficulty
	This is an introductory workshop, for people who have tried basic programming in Python before.
		
	\subsection*{How to use these booklets}

	The aim of these booklets is to help you attempt these workshops at home, and to explain concepts in more detail than at the workshop. You don't need to follow these booklets during the workshop, but you can if you'd like 0to.
	
	%Code Listings
		When you need to make changes to your code, they'll be presented in \textit{listings} like the example below. Some lines may be repeated across multiple listings, so check the line numbers to make sure you're not copying something twice.

	\lstinputlisting[style=Python, lastline=2]{McrRaspJam/015_GPIOZero/0_introduction/led.py}
	
	
	
	Occasionally, a concept will be explained in greater detail in \textit{asides}, like the one below. You can read these as you wish, but they're not required to complete the workshop.
	
\begin{aside}[Resistors]
	Resistors are most commonly used to limit the amount of current flowing through part of an electrical circuit.
	
	For example we use resistors in series with LEDs, as otherwise they could draw so much power that they destroy themselves.
	
	Buttons have almost no internal resistance, so we use high value ($\sim 10  k\Omega$) resistors to prevent current flowing straight from the power supply to ground; if we didn't, the entire CPU could be short circuited, and the Pi would lose power!
\end{aside}
	
	%\subsection*{What you'll need}
	
		% We'll be using Java for this workshop, which is installed by default in Raspbian.
	
	\subsection*{Everything else}
	
		% Aknowledgements
		These booklets were created using \textrm{\LaTeX}, an advanced typesetting system used for several sorts of books, academic reports and letters.
		\ifprint\else
			If you'd like to have a look at using LaTeX, We recommend looking at \TeX studio, which is available on most platforms, and also in the Raspbian repository.
		\fi
		
		
		% License spiel
		To allow modification and redistribution of these booklets, they are distributed under the \hbox{CC BY-SA 4.0} License.
		Latex source documents are available at \url{http://github.com/McrRaspJam/booklet-workshops}
		
		If you get stuck, find errors or have feedback about these booklets, please email me at:
		\href{mailto:jam@jackjkelly.com}{\texttt{jam@jackjkelly.com}}