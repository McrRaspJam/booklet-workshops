\section{Nested Loops} \label{sec:loops}

We've just set a block, but what if we want to set many blocks at once?

In computing, when we want to do many things repetitively, we usually use a loop. A python loop might look like this.

\begin{lstlisting}[style=Terminal, numbers=none]
>>> for i in range(1, 100):
	print(i)
1
2
3
...
\end{lstlisting}

if we did a similar thing in a Minecraft program, we could change multiple blocks at once.

\begin{lstlisting}[style=Python, title=minecraftloop.py, breaklines=true]
from mcpi.minecraft import Minecraft
mc = Minecraft.create()

pos = mc.player.getTilePos()

for i in range(0, 10):
	mc.setBlock(pos.x+i, pos.y, pos.z, 4)
\end{lstlisting}

run this program. you should now have a program that draws 10 block in a horizontal line.

A line is a one-dimensional shape. If we wanted to draw a square, we need to create a loop that works in two dimensions.

This is easy, we just need to put one loop inside another:
\begin{lstlisting}[style=Python, title=minecraftloop.py, breaklines=true, firstnumber=6]
for i in range(0, 10):
	for j in range(0, 10):
		mc.setBlock(pos.x+i, pos.y+j, pos.z, 4)
\end{lstlisting}

We've made our loop two dimensional by \textit{nesting} another loop inside it. Each time the loop runs, it creates its own loop.

This time, the program should draw a square, that extends up towards the sky.

A square is a two-dimensional shape. So, how can you change your program once more to create a three-dimensional loop and a cube of blocks?