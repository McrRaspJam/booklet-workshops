\section{Swords and TNT}

\subsection*{Making TNT Explode}

You may have noticed that whilst you can place TNT using the block ID \texttt{46}, it doesn't explode when you hit it. To make it explode, we need to tell Minecraft to spawn the special, exploding variety of the TNT block.

Each block in minecraft can have a number called a \textit{data value}. This number does something different for different blocks. Usually, it does nothing, sometimes it sets the direction a sign or staircase1 faces. For TNT, it tells the game whether the block can explode or not.

To make TNT explode, we just need to set its data value to 1.

\begin{lstlisting}[style=Python, numbers=none]
mc.setBlock(pos.x, pos.y, pos.z, 46, 1)
\end{lstlisting}

\pagebreak[1]
\subsection*{Sword Events}

The sword in \textit{Minecraft: Pi Edition} has a special feature, it records all the blocks you hit, and the API allows us to do something with each of these block hits.

This Program will make every block you hit with your sword turn into gold!

\begin{lstlisting}[style=Python, title=minecraftloop.py, breaklines=true]
from time import sleep
from mcpi.minecraft import Minecraft
mc = Minecraft.create()

while True:
	blockhits = mc.events.pollBlockHits()
	
	for hit in blockhits:
		pos = hit.pos
		mc.setBlock(pos.x, pos.y, pos.z, 41)
	
	mc.events.clearAll()
	sleep(0.1)	
\end{lstlisting}

Theres a few new things going on here, so let's break it down.

\texttt{blockhits = mc.events.pollBlockHits()} gets the list of recent block hits from the game. This list could be short or long, so we use a \texttt{for} loop to iterate through the list.

For each hit, we take the position of the hit and store it in \texttt{pos}. Then just like before, we can use this pos to set a block.

Finally, we need to tell the game to clear the events list, we only want to change the block that got hit once.

\begin{itemize}[noitemsep]
	\item how can you make the sword turn things into exploding TNT?
	\item how can you make more than one block change each time you make a hit?
	\item is there anything interesting you could do if you \textit{didn't} clear the event list?
\end{itemize}