	%Workshop Description
	The notes include a number of guides to help you get your own Raspberry Pi up and running. You should:
	
	\begin{itemize}[nosep]
		\item \textbf{Read \autoref{sec:PiEquiptment}} if you're unsure if you have everything you need to run a Raspberry Pi.
		\item \textbf{Read \autoref{sec:NOOBS}} if you want to set up a blank SD card for the Raspberry Pi.
		\item \textbf{Read \autoref{sec:setup}} to troubleshoot common problems when booting your Pi.
		\item \textbf{Read \autoref{sec:Resources}} for a collection of Raspberry Pi resource sites, where you'll find magazines, tutorials and project ideas.
	\end{itemize}
		
	\subsection*{About these booklets}
		
	%Aknowledgements
	These workshop booklets were created using {\fontfamily{rfdefault}\selectfont \LaTeX}, an advanced typesetting system that is widely used for several sorts of books, academic reports and letters.
	
	%License spiel
	To allow modification and redistribution of these booklets, they are distributed under the \hbox{CC BY-SA 4.0} License. LaTeX source documents are available at \url{http://github.com/McrRaspJam/booklet-workshops}				
	
	%File Downloads
	All of our workshop resources are available to download from a Google Drive at
\url{http://bit.ly/mcrraspjam}.
There, you can find a PDF copy of this booklet, as well as template and completed program files for each workshop.
			
	%Code Listings
	%	When you need to make changes to your code, they'll be presented in boxes like the following:

	\lstinputlisting[style=Java, firstline=1, firstnumber=1, lastline=4]{McrRaspJam/013_Objects/0_introduction/HelloWorld.java}
	
	You might not need to copy everything, so check the line numbers to make sure you're not copying something twice.
	
	%Terminal commands are listed as such, copy everything \textit{after} the dollar sign. Lines without dollar signs are example outputs, and do not need to be copied.
			
\begin{lstlisting}
$ cd code
$ python3 helloworld.py 
Hello, World!
Hello, World!
\end{lstlisting}
	%Occasionally, a concept will be explained in greater detail in \textit{asides}, like the one below. You can read these as you wish, but they're not required to complete the workshop.
	
\begin{aside}[Cryptography]
	We call the creation and study of ciphers \textit{cryptography}. `Crypto-' comes from the greek word \textit{kruptos}, meaning `hidden'.
\end{aside}
		
	\subsection*{Questions?}
		If you get stuck with any of the instructions, find errors or have feedback about these booklets, email:
		\url{jam@jackjkelly.com}\label{email}