	%Workshop Description
	The notes include a number of guides to help you get your own Raspberry Pi up and running. You should:
	
	\begin{itemize}[nosep]
		\item \textbf{Read \autoref{sec:PiEquiptment}} if you're unsure if you have everything you need to run a Raspberry Pi.
		\item \textbf{Read \autoref{sec:NOOBS}} if you want to set up a blank SD card for the Raspberry Pi.
		\item \textbf{Read \autoref{sec:setup}} to troubleshoot common problems when booting your Pi.
		\item \textbf{Read \autoref{sec:Resources}} for a collection of Raspberry Pi resource sites, where you'll find magazines, tutorials and project ideas.
	\end{itemize}
	
	\newpage
	\subsection*{About these booklets}
		
	%Aknowledgements
	These workshop booklets were created using {\fontfamily{rfdefault}\selectfont \LaTeX}, an advanced typesetting system that is widely used for several sorts of books, academic reports and letters.
	
	%License spiel
	To allow modification and redistribution of these booklets, they are distributed under the \hbox{CC BY-SA 4.0} License. LaTeX source documents are available at \url{http://github.com/McrRaspJam/booklet-workshops}				
	
	%File Downloads
	All of our workshop resources are available to download from a Google Drive at
\url{http://bit.ly/mcrraspjam}.
There, you can find a PDF copy of this booklet, as well as template and completed program files for each workshop.
			
	%Code Listings
	%	When you need to make changes to your code, they'll be presented in \textit{listings} like the example below. Some lines may be repeated across multiple listings, so check the line numbers to make sure you're not copying something twice.

	\lstinputlisting[style=Python, breaklines=true]{McrRaspJam/016_Ciphers/0_introduction/alphabet.py}
	
	
	
	%When you need to type a command in the command line interface, they will be listed like the following example. Copy everything \textit{after} the dollar sign. Lines without dollar signs are example outputs, and do not need to be copied.
			
\begin{lstlisting}[style=Terminal]
$ java HelloWorld
Hello, World!
\end{lstlisting}
	%Occasionally, a concept will be explained in greater detail in \textit{asides}, like the one below. You can read these as you wish, but they're not required to complete the workshop.
	
\begin{aside}[Resistors]
	Resistors are most commonly used to limit the amount of current flowing through part of an electrical circuit.
	
	For example we use resistors in series with LEDs, as otherwise they could draw so much power that they destroy themselves.
	
	Buttons have almost no internal resistance, so we use high value ($\sim 10  k\Omega$) resistors to prevent current flowing straight from the power supply to ground; if we didn't, the entire CPU could be short circuited, and the Pi would lose power!
\end{aside}
		
	\subsection*{Questions?}
		If you get stuck with any of the instructions, find errors or have feedback about these booklets, email:
		\url{jam@jackjkelly.com}\label{email}