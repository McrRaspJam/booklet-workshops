\section{Falling Snow}

Change your main function, so that it calls the \texttt{move()} function after drawing:

\lstinputlisting[style=python, title=main(), firstnumber=1, firstline=8, lastline=12]{McrRaspJam/008_SenseHAT/extension/code/2_movement.py}

Now we can start thinking of ways to move our snowflakes. The most basic thing they should do is fall towards the ground.

If we increase the y coordinate of each flake, every time the program loops, the flake will descend by one pixel.

\begin{lstlisting}[style=Python, title=move()]
def move():
	for snowflake in snowflakes:
		snowflake[1] += 1
\end{lstlisting}

If you run your program now, you can see your snowflakes are now moving. However, once they reach the bottom of the screen, they will cause an error.

We can't draw outside of the screen, so we'll delete each snowflake as they reach the bottom of the screen:

\begin{lstlisting}[style=Python, title=move(), breaklines=true, firstnumber=4]

		if snowflake[1] > 7:
			snowflakes.remove(snowflake)
\end{lstlisting}

You'll also need to call \texttt{hat.clear()} in your draw function to stop snowflakes from leaving `trails'.

\begin{lstlisting}[style=Python, title=draw()]
def draw():
hat.clear()
for snowflake in snowflakes:
hat.set_pixel(snowflake[0], snowflake[1], 255, 255, 255)
\end{lstlisting}

Your program should now loop continuously. Our snowflakes fall, like they would under gravity, but we should also simulate the effect of wind. How can you extend the \texttt{move()} function to make flakes move randomly left and right?

\section{Creating more snow}

Currently, each time the program loops, we create one extra flake of snow, we'll call this the \textit{spawn rate} of the snowflakes. Let's modify our program so we can set the spawn rate.

A function called \texttt{add\_snowflakes()} has been provided in a text file in the same folder as \texttt{snowflakes.py}. Copy this function into your program.

\begin{lstlisting}[style=Python, numbers=none]
def add_snowflakes(count):
	count_left = count

	while count_left >= 1:
		add_snowflake()
		count_left -= 1

	if count_left > 0:
		count = int(round(1 / count_left, 0))
		if randint(0, count) == 0:
			add_snowflake()
\end{lstlisting}

This function will call \texttt{add\_snowflake()} the correct number of times based on the spawn rate. If you call \texttt{add\_snowflakes(1)}, it will add one snowflake, \texttt{add\_snowflakes(3)} will add three.
It also works for decimal numbers, \texttt{add\_snowflakes(0.3)} will create a snowflake \textit{approximately} one third of the time.

To update your main method, simply change the \texttt{add\_snowflake()} function call to \texttt{add\_snowflakes()}, using 2 as the parameter:

\lstinputlisting[style=python, title=main(), firstnumber=1, firstline=8, lastline=10]{McrRaspJam/008_SenseHAT/extension/code/3_changerate.py}

Now, when you run your program it should be even snowier!
