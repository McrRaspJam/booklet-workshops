\section{Directories}

	\subsection{Where am I?}
	
		Once you've logged in, you should see a line that looks like this.
	
		\begin{lstlisting}[style=Terminal]
pi@raspberrypi:~$
		\end{lstlisting}
	
		This is your command prompt. Let's break up that line.
		
		\begin{itemize}[nosep]
			\item The first part is the current user. We logged in as \textbf{pi}, so that appears here.
			\item Following the \textbf{@} is the hostname, or the `machine name'. \textbf{raspberrypi} is the default hostname in Raspbian.
			\item Following the colon, your current director is shown. The tilde character \textbf{\~} is a shortcut for your home folder.
			\item \textbf{\$} is the current prompt. When this is the last thing printed, the terminal is ready to receive an instruction.
		\end{itemize}
	
		Let's confirm that the tilde is pointing at our home folder. To find the full address to our current directory, we can use the \textbf{pwd} command, which stands for `print working directory'.

		\begin{lstlisting}[style=Terminal]
pi@raspberrypi:~$ pwd
/home/pi
		\end{lstlisting}
	
	%\subsection{The Filesystem}
	
	%	\textit{<Written explanation to be added after the workshop>}
	
	\subsection{Moving around}
	
		So, we're currently in our home directory, and there are two levels above us. to move to the root folder, we can use the \textbf{cd} command, which stands for `change directory'.
		
		\begin{lstlisting}[style=Terminal]
pi@raspberrypi:~$ cd ..
pi@raspberrypi:/home $ cd ..
pi@raspberrypi:/ $ pwd
/
		\end{lstlisting}

		\textbf{..} causes us to move up a directory. to move into a directory, we can type its name.
	
		\begin{lstlisting}[style=Terminal]
pi@raspberrypi:/ $ cd usr
pi@raspberrypi:/usr $ pwd
/usr
		\end{lstlisting}

		We can type an absolute directory starting with `/', the root folder. 
		
		\begin{lstlisting}[style=Terminal]
pi@raspberrypi:/usr $ cd /media
pi@raspberrypi:/media $ pwd
/media
		\end{lstlisting}

Using the cd command without a directory or .. takes us to the home folder. We can string together directory changes using forward slashes.

		\begin{lstlisting}[style=Terminal]
pi@raspberrypi:/media $ cd
pi@raspberrypi:~ $ pwd
/home/pi
pi@raspberrypi:~ $ cd ../../usr
pi@raspberrypi:/usr $ pwd
/usr
		\end{lstlisting}

		Now have a go at finding at finding the location of the Python 3 binary executable. You can use the \textbf{whereis} command to find the location of a program.
		
		\begin{lstlisting}[style=Terminal]
$ whereis python3
		\end{lstlisting}

		The first part of the output will be the directory of the binary file.