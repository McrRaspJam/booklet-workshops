\section{Getting started}

Boot your Raspberry Pi to the desktop, and open IDLE 3 or your preferred editor. Open the template Python file \texttt{snowflakes.py}. The main section of the program will currently look like this:

\lstinputlisting[style=python, title=snowflakes.py, numbers=none, firstline=6, lastline=22]{McrRaspJam/008_SenseHAT/extension/code/template.py}

The program has a main function, but we can see that the program doesn't yet do anything, as it is calling functions which are empty.

\subsection*{Create a snowflake}

Let's begin by creating the \texttt{add\_snowflake()} function. We will store the position of each snowflake on the screen, so to create a new snowflake, we can create a coordinate pair like so:

\begin{lstlisting}[style=Python]
def add_snowflake():
	newflake = [0, 0]
\end{lstlisting}

We'll now store this new `flake' in the variable \texttt{snowflakes}, which has been created at the top of our program.

\begin{lstlisting}[style=Python, firstnumber=3]
	snowflakes.append(newflake)
\end{lstlisting}

If you run your program now, a snowflake is stored in the \texttt{snowflakes} variable. Type \texttt{snowflakes} into the Python Shell, and you should see [0,0] in its contents.

\subsection*{Draw the snowflake}

Our snowflake now exists in our program, but is not currently being drawn. You may recall from the workshop that the Sense HAT command for drawing a pixel is:

\begin{lstlisting}[style=Python, numbers=none]
hat.set_pixel(x, y, R, G, B)
\end{lstlisting}

Our program will soon have many snowflakes, so we need to loop through them like so:

\begin{lstlisting}[style=Python, title=draw()]
def draw():
	for snowflake in snowflakes:
\end{lstlisting}

then, you can draw each snowflake like so:

\begin{lstlisting}[style=Python, breaklines=true, firstnumber=3]
		hat.set_pixel(snowflake[0], snowflake[1], 255, 255, 255)	
\end{lstlisting}

Your program should now draw a white pixel at the coordinate (0,0), the top left of the LED matrix.

\subsection*{More Snowflakes}

Let's start adding more snowflakes. First, we need to make the program loop, so it keeps adding more snowflakes:

\lstinputlisting[style=python, title=main(), firstline=8, lastline=11]{McrRaspJam/008_SenseHAT/extension/code/1_createflakes.py}

Add a sleep command, to stop the program running too quickly.

\begin{lstlisting}[style=Python, breaklines=true, firstnumber=5]
		sleep(0.2)	
\end{lstlisting}

Currently, our program only creates snowflakes at one position, (0, 0) so we should change the \texttt{add\_snowflake()} function to place them randomly.

The \texttt{randint()} function from the \texttt{Random} library will produce a random whole number between a specified range. If we set the x coordinate to a random number between 0 and 7, snowflakes will appear across the whole top row of the screen:

\lstinputlisting[style=python, title=add\_snowflake(), firstline=29, lastline=31]{McrRaspJam/008_SenseHAT/extension/code/1_createflakes.py}

Try running your program again. The top row should soon fill up, but our flakes aren't falling just yet!
