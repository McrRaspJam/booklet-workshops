\setcounter{section}{-1}
\section{Introduction}

	Included in Raspbian, the operating system we run on our Raspberry Pi's is a copy of the `Pi Edition' of \textit{Minecraft}. 
	
	This version of Minecraft is based on the creative mode from \textit{Minecraft: Pocket Edition}, but in this version we can use the Python programming language to modify the game world.
	
	%Difficulty
	This is an introductory workshop. It's useful to have programmed in Python before,  but all of the programming concepts will be covered from scratch.
		
	\subsection*{How to use these booklets}

	The aim of these booklets is to help you attempt these workshops at home, and to explain concepts in more detail than at the workshop. You don't need to refer to use these booklets during the workshop, but you can if you'd like to.
	
	%Code Listings
		When you need to make changes to your code, they'll be presented in \textit{listings} like the example below. Some lines may be repeated across multiple listings, so check the line numbers to make sure you're not copying something twice.

	\lstinputlisting[style=Python, title=helloworld.py]{McrRaspJam/017_Minecraft/0_introduction/helloworld.py}
	
	
	%Occasionally, a concept will be explained in greater detail in \textit{asides}, like the one below. You can read these as you wish, but they're not required to complete the workshop.
	
\begin{aside}[Cryptography]
	We call the creation and study of ciphers \textit{cryptography}. `Crypto-' comes from the greek word \textit{kruptos}, meaning `hidden'.
\end{aside}

	\subsection*{What you'll need}
		All the software you need for this workshop is pre-installed on recent versions of Raspbian.
	\subsection*{Everything else}
	
		% Aknowledgements
		These booklets were created using \textrm{\LaTeX}, an advanced typesetting system used for several sorts of books, academic reports and letters.
			
		If you'd like to have a look at using LaTeX, We recommend looking at \TeX studio, which is available on most platforms, and also in the 	Raspbian repository.
		
		% License spiel
		To allow modification and redistribution of these booklets, they are distributed under the \hbox{CC BY-SA 4.0} License.
		Latex source documents are available at \url{http://github.com/McrRaspJam/booklet-workshops}
		
		If you get stuck, find errors or have feedback about these booklets, please email me at:
		\href{mailto:jam@jackjkelly.com}{\texttt{jam@jackjkelly.com}}