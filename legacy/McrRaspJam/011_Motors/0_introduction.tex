\setcounter{section}{-1}
\section{Introduction}
	
	
	%Workshop Description	
	The workshop shows how to control a motor with the Raspberry Pi, using a breadboard circuit and a standard motor controller chip. The motor is then programmed with the Python GPIO library.

	%Difficulty
	This is an intermediate workshop, it will help to have some previous experience with Python.
		
	%Aknowledgements
	These booklets were created using {\fontfamily{rfdefault}\selectfont \LaTeX}, an advanced typesetting system used for several sorts of books, academic reports and letters. The schematic diagrams were created using \textit{Fritzing}, a free hardware prototyping and PCB design software.
		
	%License spiel
	To allow modification and redistribution of these booklets, they are distributed under the \hbox{CC BY-SA 4.0} License. LaTeX source documents are available at \url{http://github.com/McrRaspJam/booklet-workshops}
	
	
	\subsection*{What you'll need}
		
			
		The hardware requirements for this workshop are listed below.
	
		\begin{itemize}[nosep]		
			\item A solderless breadboard, preferably one with separate bus strips. (those labelled + \& -)
			\item Male-to-female \& male-to-male jumper cables.
			\item A 6V DC motor.
			\item A L293D or L293DNE motor controller.
			\item (Optional) a 4$\times$AA Battery Pack				
		\end{itemize}

		For the LED exercise in \autoref{sec:LED}, you will also require:
			
		\begin{itemize}[nosep]	
			\item An LED
			\item A $\sim 330 \Omega$ Resistor
		\end{itemize}	
			
	%Code Listings
	\subsection*{Code listings \& asides}
	
	When you need to make changes to your code, they'll be presented in boxes like the following:

	\lstinputlisting[style=Python]{McrRaspJam/011_Motors/0_introduction/helloworld.py}
	
	You might not need to copy everything, so check the line numbers to make sure you're not copying something twice.
		
	\subsection*{Questions?}
		If you get stuck, find errors or have feedback about these booklets, email:
		\url{jam@jackjkelly.com}\label{email}