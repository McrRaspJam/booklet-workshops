\section{Manipulating the World}
\label{sec:manipulating_the_world}

	\subsection{Chat}

		Minecraft Pi Edition provides a simple function for posting messages into the game's chat called \texttt{postToChat}. This function allows us to pass in some text and have the text appear in-game.

		\lstinputlisting[style=Python, numbers=none, breaklines=true]{McrRaspJam/014b_MinecraftSenseHAT/2_manipulating_the_world/chat.py}

		\textbf{Extra:} Why not try changing what is said in chat or even try changing how many times it is said?

	\subsection{Positions}

		Before we start placing blocks in the world, we need to know where we're placing them! Positions in Minecraft are represented by three numbers, \texttt{x}, \texttt{y} and \texttt{z}. The \texttt{x} and \texttt{z} values are your horizontal position relative to the centre of the world and the \texttt{y} value is how high up you are from the bottom of the world.

		You can get the player's current position using \texttt{player.getPos()}, which returns an object containing the player's current position. This can be useful for moving the player around using code or modifying blocks relative to the player.

		\lstinputlisting[style=Python, numbers=none, breaklines=true]{McrRaspJam/014b_MinecraftSenseHAT/2_manipulating_the_world/position.py}

	\subsection{Blocks}

		The Minecraft world consists of a set of blocks, each with their own ID which represents the type of block.

		\begin{center}
			\begin{tabular}{ l l }
				\hline
				ID & Type \\
				\hline
				\hline
				0 & Air \\
				1 & Stone \\
				2 & Grass \\
				3 & Dirt \\
				4 & Cobblestone \\
				5 & Wooden Planks \\
				\multicolumn{2}{c}{...} \\
				247 & Nether Reactor Core \\
				\hline
			\end{tabular}
		\end{center}

		There are lots of types of blocks in Minecraft and it would be a pain to have to look up their IDs every time you want to place a block. This is where the Minecraft API comes in again, as it provides us a list of block names and their respective IDs.

		\lstinputlisting[style=Python, numbers=none, breaklines=true]{McrRaspJam/014b_MinecraftSenseHAT/2_manipulating_the_world/block.py}

		Note that we imported \texttt{block} from \texttt{mcpi} as opposed to \texttt{minecraft}. If you are looking to use both Minecraft and the blocks (as you would do if you were writing a normal program), you can import them both by separating their names by a comma.

		\lstinputlisting[style=Python, numbers=none, breaklines=true]{McrRaspJam/014b_MinecraftSenseHAT/2_manipulating_the_world/import.py}

		We can place and remove blocks using \texttt{setBlock} and specifying the position and type of block to place. An easy way to try out placing blocks is by placing a block over the player's head.

		\lstinputlisting[style=Python, numbers=none, breaklines=true]{McrRaspJam/014b_MinecraftSenseHAT/2_manipulating_the_world/block_head.py}

		\textbf{Extra:} Why not try changing the block from grass to a different block?

		\textbf{Extra:} How would you place a block beneath the player instead of above?

		\textbf{Extra:} How would you keep placing blocks wherever the player moves? \textit{Hint: Loops allow us to run the same lines of code more than once.}

		\begin{aside}[Hitting Blocks]
			Block hit events are only emitted by the game when the player \textit{right clicks} a block with a sword, so if your code doesn't seem to be working make sure you're using a sword!
		\end{aside}

	\subsection{Events}

		Events are actions that the player has performed, such as hitting a block or jumping in the air. We can watch what events are happening from Python and act upon them in our code. To do this we need a loop that keeps checking for events and some code to handle the event that is happening.

		In this section we'll be writing some code to turn everything the player touches into gold!

		Firstly we need the basic framework for handling events, which consists of a loop that runs forever and a function that gets a list of all of the events that we have not processed yet.

		\lstinputlisting[style=Python, numbers=none, breaklines=true]{McrRaspJam/014b_MinecraftSenseHAT/2_manipulating_the_world/event_framework.py}

		This code prints out the position of the block any time the player hits a block, which we can use to turn that block into gold! We can use the function \texttt{setBlock} that we learned about earlier to change the block the player hit into a gold block.

		\lstinputlisting[style=Python, numbers=none, breaklines=true]{McrRaspJam/014b_MinecraftSenseHAT/2_manipulating_the_world/midas.py}

		Our final code simply takes the position of the block that was hit and turns it into a gold block.

		\textbf{Extra:} Why not try changing the block from gold to a different block?

		\textbf{Extra:} How would you make it so that the block that was clicked is removed instead of replaced? \textit{Hint: Every empty block in the world is actually an air block.}
