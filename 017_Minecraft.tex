%
% Manchester Raspberry Jam Workshop Booklet
%
% This template is designed to try and aide consistency between each booklet.
% Insert files as instructed by comments, things like tables of contents are created by other linked files
%

% Is this document PRINT or WEB format?
% (you can use the toggle to cut content for paper, see etoolbox)
\newif\ifprint
\printtrue

% Page Formatting
\ifprint
	\documentclass[a4paper, twocolumn, twoside, 10pt]{article}
	\usepackage[margin=2cm]{geometry}
	\setlength{\columnsep}{1.5cm}
	\setlength{\parskip}{12pt}
\else
	\documentclass[a4paper, onecolumn, oneside, 11pt]{article}
	\usepackage[margin={4cm, 3cm}]{geometry}
	\setlength{\parskip}{12pt}
\fi



% FONT and text format
\usepackage[utf8x]{inputenc}	
\usepackage[UKenglish]{babel}
\ifprint
	% Print version now use sans-serif font
	\usepackage[T1]{fontenc}
	\usepackage[sfdefault]{roboto}
	\usepackage[T1]{fontenc}
	
	\usepackage[usenames, dvipsnames]{color}				%Font Colour
	\usepackage[colorlinks=false]{hyperref}	%URLs
\else
	\usepackage[T1]{fontenc}
	\usepackage[sfdefault]{roboto}
	\usepackage[T1]{fontenc}
	\usepackage[usenames, dvipsnames]{color}				%Font Colour
	\usepackage[colorlinks=true,linkcolor=black,urlcolor=WildStrawberry]{hyperref}
\fi



% Table of Contents Format
\usepackage{tocloft}
\addtocontents{toc}{\cftpagenumbersoff{subsection}}
\setcounter{tocdepth}{2}
\setcounter{secnumdepth}{2}



% Section spacing
\ifprint
	\usepackage[compact]{titlesec}
\fi



% Listings and Asides
\usepackage{McrRaspJam/common/listings}
\usepackage{McrRaspJam/common/asides}



%Clear page for web version
\newcommand{\webclearpage}{
	\ifprint
	\else
		\clearpage
	\fi
}

% misc.
\usepackage{enumitem}									%List spacing changes
\usepackage[toc,page]{appendix}							%Appendix package
\usepackage{graphicx}									% TOC?
\usepackage{etoolbox}									%Boolean used for print/web switching
\usepackage{fancyvrb}									%Centered verbatim
\usepackage{amsmath}									%Aligned maths

\setlength\tabcolsep{4 pt}								%Reduce cell width in tabulars

% Enter the document title here
\newcommand{\workshopTitle}{Workshop 17: \textit{Minecraft: Pi Edition}}

% Enter the author of this workshop
\newcommand{\workshopAuthor}{Jack Kelly}


\begin{document}
	% Title Format
\ifprint
	\title{Manchester Raspberry Jam \\ \workshopTitle}
	\author{}
\else
	\title{
		\begin{center}
			\includegraphics[width=30mm]{McrRaspJam/common/logo-512}
		\end{center}
		\vspace{12pt}
		\workshopTitle
	}
	\author{
		\workshopAuthor
	}
\fi

\date{\vspace{-16pt}}
\maketitle


% Online download location
\ifprint
	\begin{mdframed}[rightline=false, leftline=false]
		\scriptsize
		This booklet is available online at \mbox{\href{https://drive.google.com/open?id=0B_1SFjX_5JrmfnhpX0pPRXl6bmJNal8zdUxMeWZOdjJyZVdzU3V6UnBGdlVIMENtbFFkbVk}{bit.ly/McrRaspJam}}
		\normalsize
	\end{mdframed}
\fi
	
	%Place a SINGLE paragraph summary here
	This workshop teaches the basics of the Python API for \textit{Minecraft: Pi Edition}.
	
	%Difficulty
	\textit{Difficulty: Beginner workshop}

	\ifprint
		\renewcommand{\baselinestretch}{0.75}\normalsize
		\tableofcontents
		\renewcommand{\baselinestretch}{1.0}\normalsize
	\else
		\tableofcontents
	\fi
	
	%
	% Input the main CONTENT below, sans title page or contents.
	% Recommend inputting per section, and adding page breaks here.
	%
	% \webclearpage command is provided, will break page for web format only.
	%
	
	\setcounter{section}{-1}
\section{Introduction}

	Included in Raspbian, the operating system we run on our Raspberry Pi's is a copy of the `Pi Edition' of \textit{Minecraft}. 
	
	This version of Minecraft is based on the creative mode from \textit{Minecraft: Pocket Edition}, but in this version we can use the Python programming language to modify the game world.
	
	%Difficulty
	This is an introductory workshop. It's useful to have programmed in Python before,  but all of the programming concepts will be covered from scratch.
		
	\subsection*{How to use these booklets}

	The aim of these booklets is to help you attempt these workshops at home, and to explain concepts in more detail than at the workshop. You don't need to refer to use these booklets during the workshop, but you can if you'd like to.
	
	%Code Listings
		When you need to make changes to your code, they'll be presented in \textit{listings} like the example below. Some lines may be repeated across multiple listings, so check the line numbers to make sure you're not copying something twice.

	\lstinputlisting[style=Python, title=helloworld.py]{McrRaspJam/017_Minecraft/0_introduction/helloworld.py}
	
	
	%Occasionally, a concept will be explained in greater detail in \textit{asides}, like the one below. You can read these as you wish, but they're not required to complete the workshop.
	
\begin{aside}[Cryptography]
	We call the creation and study of ciphers \textit{cryptography}. `Crypto-' comes from the greek word \textit{kruptos}, meaning `hidden'.
\end{aside}

	\subsection*{What you'll need}
		All the software you need for this workshop is pre-installed on recent versions of Raspbian.
	\subsection*{Everything else}
	
		% Aknowledgements
		These booklets were created using \textrm{\LaTeX}, an advanced typesetting system used for several sorts of books, academic reports and letters.
			
		If you'd like to have a look at using LaTeX, We recommend looking at \TeX studio, which is available on most platforms, and also in the 	Raspbian repository.
		
		% License spiel
		To allow modification and redistribution of these booklets, they are distributed under the \hbox{CC BY-SA 4.0} License.
		Latex source documents are available at \url{http://github.com/McrRaspJam/booklet-workshops}
		
		If you get stuck, find errors or have feedback about these booklets, please email me at:
		\href{mailto:jam@jackjkelly.com}{\texttt{jam@jackjkelly.com}}
		\clearpage
		
	\section{Python \& Minecraft Basics}

	Boot your Raspberry Pi to the desktop. The default login is as follows, if your pi doesn't log in automatically.
	
	\begin{tabular}{rl}
		\textbf{username} & pi \\ 
		\textbf{password} & raspberry
	\end{tabular} 
	
	Once at the desktop, Open IDLE from the application menu under \textbf{Programming $\rightarrow$ Python 3 (IDLE)}. The first window that will open is the Python `Shell' window, where our programs will run.
	
	From the shell window press \textbf{File $\rightarrow$ New File} to open a second window. This is where we will write our programs.
	
	\subsection*{Hello, World!}
	
		The first program we write in a new programming language is "Hello, World!". In python, this takes just one line:
		
		\begin{lstlisting}[style=Python, title=helloworld.py]
print("Hello, World!")
		\end{lstlisting}
		
		We can now run our program by clicking \textbf{Run $\rightarrow$ Run Module}, or pressing \textbf{F5}. After saving the file, your program should then run in the shell window.
		
		\begin{lstlisting}[style=Terminal, numbers=none]
=========== RESTART: ===========
Hello, World!
>>> 
		\end{lstlisting}
	
	\pagebreak[1]
	\subsection*{Loading the Minecraft API}
	
		We need to tell Python that this is a Minecraft program. To do this, we start all of our Minecraft programs with these two lines:
	
		\begin{lstlisting}[style=Python, title=hellominecraft.py]
from mcpi.minecraft import Minecraft
mc = Minecraft.create()
		\end{lstlisting}
	
	\pagebreak[1]
	\subsection*{Hello, Minecraft!}
		
		Let's now do the same program in Minecraft, by posting text to the in-game chat.

		If we use \texttt{print()} like we did in the last program, the text would still appear in the shell window. To post to Minecraft, we need to use one of the API functions:	

		\begin{lstlisting}[style=Python, firstnumber=3]

mc.postToChat("Hello, Minecraft!")
		\end{lstlisting}
		
		You can run your program again, but first, make sure Minecraft is running and in-game, otherwise you'll get a an error.
		
	\pagebreak[1]
	\subsection*{Teleportation}
	
		In computing, we keep bits of data in things called \textit{variables}. This could be a number, some text, or something more complex, like the position of a player in a videogame.
		
		In python, we might use variables as follows:
		
		\begin{lstlisting}[style=Terminal, numbers=none]
>>> a = 3
>>> b = 5
>>> a + b
8
		\end{lstlisting}
		
		for our program, we'll get the player position, and place it in a variable called \texttt{pos}.
		
		\begin{lstlisting}[style=Python, firstnumber=5]
		
pos = mc.player.getTilePos()
		\end{lstlisting}
		
		\texttt{pos} is a collection of three numbers, which can be accessed using \texttt{pos.x}, \texttt{pos.y} and \texttt{pos.z}, we can teleport the player by using

		\begin{lstlisting}[style=Python, firstnumber=7]
mc.player.setPos(pos.x, pos.y+100, pos.z)
		\end{lstlisting}
		
		What happens when you run the program? Try using different x, y and z values and see what happens.
		
	\pagebreak[1]	
	\subsection*{Placing Blocks}
		
		to place a block at a set of coordinates, we can do:
		
		\begin{lstlisting}[style=Python, firstnumber=8]

mc.setBlock(pos.x, pos.y, pos.z, 4)
		\end{lstlisting}
		
		this will turn a block into cobblestone. To change the type of block, change the number 4 into one of the block IDs listed in appendix \autoref{sec:blockids}. 
		
		Removing a block is simple, you just need to set the block ID to air! (0)
		
		
		\webclearpage
		
	\section{Nested Loops} \label{sec:loops}

We've just set a block, but what if we want to set many blocks at once?

In computing, when we want to do many things repetitively, we usually use a loop. A python loop might look like this.

\begin{lstlisting}[style=Terminal, numbers=none]
>>> for i in range(1, 100):
	print(i)
1
2
3
...
\end{lstlisting}

if we did a similar thing in a Minecraft program, we could change multiple blocks at once.

\begin{lstlisting}[style=Python, title=minecraftloop.py, breaklines=true]
from mcpi.minecraft import Minecraft
mc = Minecraft.create()

pos = mc.player.getTilePos()

for i in range(0, 10):
	mc.setBlock(pos.x+i, pos.y, pos.z, 4)
\end{lstlisting}

run this program. you should now have a program that draws 10 block in a horizontal line.

A line is a one-dimensional shape. If we wanted to draw a square, we need to create a loop that works in two dimensions.

This is easy, we just need to put one loop inside another:
\begin{lstlisting}[style=Python, title=minecraftloop.py, breaklines=true, firstnumber=6]
for i in range(0, 10):
	for j in range(0, 10):
		mc.setBlock(pos.x+i, pos.y+j, pos.z, 4)
\end{lstlisting}

We've made our loop two dimensional by \textit{nesting} another loop inside it. Each time the loop runs, it creates its own loop.

This time, the program should draw a square, that extends up towards the sky.

A square is a two-dimensional shape. So, how can you change your program once more to create a three-dimensional loop and a cube of blocks?
		\webclearpage
	
	\section{Swords and TNT}

\subsection*{Making TNT Explode}

You may have noticed that whilst you can place TNT using the block ID \texttt{46}, it doesn't explode when you hit it. To make it explode, we need to tell Minecraft to spawn the special, exploding variety of the TNT block.

Each block in minecraft can have a number called a \textit{data value}. This number does something different for different blocks. Usually, it does nothing, sometimes it sets the direction a sign or staircase1 faces. For TNT, it tells the game whether the block can explode or not.

To make TNT explode, we just need to set its data value to 1.

\begin{lstlisting}[style=Python, numbers=none]
mc.setBlock(pos.x, pos.y, pos.z, 46, 1)
\end{lstlisting}

\pagebreak[1]
\subsection*{Sword Events}

The sword in \textit{Minecraft: Pi Edition} has a special feature, it records all the blocks you hit, and the API allows us to do something with each of these block hits.

This Program will make every block you hit with your sword turn into gold!

\begin{lstlisting}[style=Python, title=minecraftloop.py, breaklines=true]
from time import sleep
from mcpi.minecraft import Minecraft
mc = Minecraft.create()

while True:
	blockhits = mc.events.pollBlockHits()
	
	for hit in blockhits:
		pos = hit.pos
		mc.setBlock(pos.x, pos.y, pos.z, 41)
	
	mc.events.clearAll()
	sleep(0.1)	
\end{lstlisting}

Theres a few new things going on here, so let's break it down.

\texttt{blockhits = mc.events.pollBlockHits()} gets the list of recent block hits from the game. This list could be short or long, so we use a \texttt{for} loop to iterate through the list.

For each hit, we take the position of the hit and store it in \texttt{pos}. Then just like before, we can use this pos to set a block.

Finally, we need to tell the game to clear the events list, we only want to change the block that got hit once.

\begin{itemize}[noitemsep]
	\item how can you make the sword turn things into exploding TNT?
	\item how can you make more than one block change each time you make a hit?
	\item is there anything interesting you could do if you \textit{didn't} clear the event list?
\end{itemize}
	\webclearpage
		
	\begin{appendices}
		
		\section{Minecraft Block IDs} \label{sec:blockids}
			
			\scriptsize
			\begin{tabular}{rl}
				AIR & 0 \\
				STONE & 1 \\
				GRASS & 2 \\
				DIRT & 3 \\
				COBBLESTONE & 4 \\
				WOOD\_PLANKS & 5 \\
				SAPLING & 6 \\
				BEDROCK & 7 \\
				WATER\_FLOWING & 8 \\
				WATER & 8 \\
				WATER\_STATIONARY & 9 \\
				LAVA\_FLOWING & 10 \\
				LAVA & 10 \\
				LAVA\_STATIONARY & 11 \\
				SAND & 12 \\
				GRAVEL & 13 \\
				GOLD\_ORE & 14 \\
				IRON\_ORE & 15 \\
				COAL\_ORE & 16 \\
				WOOD & 17 \\
				LEAVES & 18 \\
				GLASS & 20 \\
				LAPIS\_LAZULI\_ORE & 21 \\
				LAPIS\_LAZULI\_BLOCK & 22 \\
				SANDSTONE & 24 \\
				BED & 26 \\
				COBWEB & 30 \\
				GRASS\_TALL & 31 \\
				WOOL & 35 \\
				FLOWER\_YELLOW & 37 \\
				FLOWER\_CYAN & 38 \\
				MUSHROOM\_BROWN & 39 \\
				MUSHROOM\_RED & 40 \\
				GOLD\_BLOCK & 41 \\
				IRON\_BLOCK & 42 \\
				STONE\_SLAB\_DOUBLE & 43 \\
				STONE\_SLAB & 44 \\
				BRICK\_BLOCK & 45 \\
				TNT & 46 \\
				BOOKSHELF & 47 \\
				MOSS\_STONE & 48 \\
				OBSIDIAN & 49 \\
				TORCH & 50 \\
				FIRE & 51 \\
				STAIRS\_WOOD & 53 \\
				CHEST & 54 \\
				DIAMOND\_ORE & 56 \\
				DIAMOND\_BLOCK & 57 \\
				CRAFTING\_TABLE & 58 \\
				FARMLAND & 60 \\
				FURNACE\_INACTIVE & 61 \\
				FURNACE\_ACTIVE & 62 \\
				DOOR\_WOOD & 64 \\
				LADDER & 65 \\
				STAIRS\_COBBLESTONE & 67 \\
				DOOR\_IRON & 71 \\
				REDSTONE\_ORE & 73 \\
				SNOW & 78 \\
				ICE & 79 \\
				SNOW\_BLOCK & 80 \\
				CACTUS & 81 \\
				CLAY & 82 \\
				SUGAR\_CANE & 83 \\
				FENCE & 85 \\
				GLOWSTONE\_BLOCK & 89 \\
				BEDROCK\_INVISIBLE & 95 \\
				STONE\_BRICK & 98 \\
				GLASS\_PANE & 102 \\
				MELON & 103 \\
				FENCE\_GATE & 107 \\
				GLOWING\_OBSIDIAN & 246 \\
				NETHER\_REACTOR\_CORE & 247
			\end{tabular}
			\normalsize

		\section{Other useful functions}

			\begin{lstlisting}[style=Python, numbers=none]
mc.getBlock(x, y, z)
			\end{lstlisting}
			\vspace{-8pt}
			Get a block ID number for the block at position (x,y,z).
			\vspace{8pt}
		
			\begin{lstlisting}[style=Python, numbers=none]
mc.setBlocks(x0, y0, z0, x1, y1, z1, id, [data])
			\end{lstlisting}			
			\vspace{-8pt}
			Place multiple blocks within two coordinates. shortcut for the nested loops from \autoref{sec:loops}.
			\vspace{8pt}
			
			\begin{lstlisting}[style=Python, numbers=none]
mc.getHeight(x, z)
			\end{lstlisting}
			\vspace{-8pt}
			Get the y coordinate (height) of the highest non-air block at that position.
			\vspace{8pt}
			
			\begin{lstlisting}[style=Python, numbers=none]
mc.saveCheckpoint()
			\end{lstlisting}
			\vspace{-8pt}
			Save the world state.
			\vspace{8pt}
			
			\begin{lstlisting}[style=Python, numbers=none]
mc.restoreCheckpoint()
			\end{lstlisting}
			\vspace{-8pt}
			Restore the world state from when it was last saved.
			\vspace{8pt}
			

\iffalse			
		\section{Function quick-reference}
		
		for a full listing see \href{http://www.stuffaboutcode.com/p/minecraft-api-reference.html}{www.stuffaboutcode.com/p/minecraft-api-reference.html}
			
			\begin{lstlisting}[style=Python, numbers=none]
mc.getBlock(x, y, z)
			\end{lstlisting}
		
			\begin{lstlisting}[style=Python, numbers=none]
mc.getBlockWithData(x, y, z)
			\end{lstlisting}
			
			\begin{lstlisting}[style=Python, numbers=none]
mc.setBlock(x, y, z, id, [data])
			\end{lstlisting}
\fi
		
	\end{appendices}
	
	%\section{Java quick start}

	In your preferred editor, create a new file and copy the following code.
	
	\lstinputlisting[style=Java]{McrRaspJam/013_Objects/1_java/HelloWorld.java}
	
	Save this file as \textbf{HelloWorld.java}, then open a terminal to the same folder, and type the following commands:
	
	\begin{lstlisting}[style=Terminal]
$ javac HelloWorld.java
$ java HelloWorld
Hello, World!
	\end{lstlisting}
	
	Now, compare that to the equivalent Python program:
	
	\begin{lstlisting}[style=Python]
	print("Hello, World")
	\end{lstlisting}
	
	Coming from most over languages, Java may seem quite complicated. You'll find out that in Java, we have to use a lot of terms that would be optional or excluded in other languages. Java is a language where seemingly nothing is implied.
	
	Luckily, for most basic programs, those lines are all the same, so it's more a matter of memorising something than anything truly complicated.
	
	\subsection{title}
	
	
	
	
	%	\webclearpage
	
\end{document}