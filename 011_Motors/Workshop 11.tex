%Layout
	\documentclass[a4paper, twocolumn, twoside, 11pt]{article}
	\setlength{\columnsep}{1cm}
	\setlength{\parskip}{6pt}
	%\usepackage{geometry}
	\usepackage[margin=1.5cm]{geometry}
	\usepackage{float}
	%\usepackage{multicol}
	\usepackage{tocloft}
	\renewcommand\cftsecafterpnum{\vskip-5pt}
	%\usepackage{hyperref}
	\usepackage{enumitem}
	\usepackage[toc,page]{appendix}

%fonts
	\usepackage[T1]{fontenc}
	\usepackage[sfdefault]{roboto}
	\usepackage[T1]{fontenc}
	\usepackage[usenames, dvipsnames]{color}
	
%URLs
	\usepackage[colorlinks=true,
				linkcolor=black,
				urlcolor=WildStrawberry]{hyperref}

%Code listings
	
	\usepackage{listings}
	\definecolor{code}{rgb}{0.95,0.95,0.95}
	\definecolor{codeframe}{rgb}{0.9,0.9,0.9}
	
	\lstset{
		frame=single,
		numbers=left,
		numbersep=10pt,
		tabsize=8,
		basicstyle=\footnotesize\ttfamily,
		rulecolor=\color{codeframe},
		backgroundcolor=\color{code}
		}

	%Python Specific
	\definecolor{pcomment}{RGB}{207,0,19}
	\definecolor{pkey}{RGB}{241,117,34 }
	\definecolor{pcode}{rgb}{0.95,0.95,1.0}
	\definecolor{pcodeframe}{rgb}{0.8,0.8,1.0}
	\definecolor{codenumber}{rgb}{0.5,0.5,0.5}
	
	\lstdefinestyle{ARM}{
		language=[ARM]{Assembler},
		rulecolor=\color{pcodeframe},
		backgroundcolor=\color{pcode},
		keywordstyle=\color{pkey},
		commentstyle=\color{pcomment},
		numberstyle=\tiny\color{codenumber},
		emphstyle=\color{purple},
		showstringspaces=false,
		tabsize=4,
		breaklines=true
		}
%Asides
	\usepackage{mdframed}
	\newenvironment{aside}
		{
		\begin{mdframed}[
			style=0,%
			leftline=false,
			rightline=false,
			innerbottommargin=2pt,
			innerleftmargin=12pt,
			innerrightmargin=0pt,
			linewidth=0.75pt,
			skipabove=6pt,
			skipbelow=6pt
			]
				\small
				\color{JungleGreen}
				\setlength{\parskip}{2pt}
				\vspace{2pt} %correct for parskip
		}
		{
		\end{mdframed}
		}

%Everything else
	\usepackage[utf8]{inputenc}
	\usepackage{graphicx}
	
%	-	-	-	-	-	-	-	-	-	-	-	-	-	-	-	-	-	-	-	-

\begin{document}

	\author{Manchester Raspberry Jam}
	\title{Workshop 11: GPIO \& Motors}
	\date{}

	\maketitle
	
	\setcounter{tocdepth}{1}
	\tableofcontents
	
	%	-	PART 0
	\setcounter{section}{-1}
	\section{Introduction}
	
		The workshop shows how to control a motor with the Raspberry Pi, using a breadboard circuit and a standard motor controller chip. The motor is then programmed with the Python GPIO library.

		This is and introductory workshop and is suitable for anybody to attempt.
		
		These booklets were created using {\fontfamily{rfdefault}\selectfont \LaTeX}, an advanced typesetting system used for several sorts of books, academic reports and letters. The schematic diagrams were created using \textit{Fritzing}, a free hardware prototyping and PCB design software.
	
		\subsection*{What you'll need}
			
			The hardware requirements for this workshop are listed below. %Illustrations are provided on page \textbf{REF} below.
	
			\begin{itemize}[noitemsep]		
				\item A solderless breadboard, preferably one with seperate bus strips. (those labelled + \& -)
				\item Male-to-female \& male-to-male jumper cables.
				\item A 6V DC motor.
				\item A L293D or L293DNE motor controller.
				\item (Optional) a 4$\times$AA Battery Pack				
			\end{itemize}
			
			For the LED exercise in \autoref{sec:LED}, you will also require:
			
			\begin{itemize}[noitemsep]	
				\item An LED
				\item A $\sim 330 \Omega$ Resistor
			\end{itemize}
			
			\newpage
				
			You can download both the template and completed python scripts for this workshop at \url{bit.ly/mcrraspjam}, along with a PDF copy of these notes.
			
			All required APIs are pre-installed with Raspbian.
			
	
		\subsection*{Code listings \& asides}
	
			When you need to make changes to your code, they'll be presented in boxes like the following:

			\lstinputlisting[style=Python]{code/helloworld.py}
	
			You might not need to copy everything, so check the line numbers to make sure you're not copying something twice.
			
			Terminal commands are listed as such, copy everything \textit{after} the dollar sign. Lines without dollar signs are example outputs, and do not need to be copied.
			
			\begin{lstlisting}
$ cd code
$ python3 helloworld.py 
Hello, World!
Hello, World!
..
			\end{lstlisting}
		
		\subsection*{Questions?}
		
			Send any corrections, suggestions or questions to:
			\url{jam@jackjkelly.com}\label{email}
	
	
	%Main Sections	
 	\newpage
	\input{section_LED}
	\input{section_motor}
	%\newpage
	
	\begin{appendices}
		
	%Don't list each appendix in Table of Contents
	\addtocontents{toc}{\setcounter{tocdepth}{-1}}	
		
	\section{Completed Code Listings}
	
		\subsection*{0\_LED.py}
			\lstinputlisting[style=Python]{code/0_LED.py}
		\newpage
		\subsection*{1\_Motor.py}
			\lstinputlisting[style=Python]{code/2_Motor.py}	
			
	\section{GPIO Pinout (GPIO.BOARD)}
	
	\begin{figure}[h]
		\label{sec:pinout}
		\centering
		\includegraphics[width=0.5\linewidth]{img/pinout_pi}
		\scriptsize
		\\Source: \url{hobbytronics.co.uk/raspberry-pi-gpio-pinout}
		\normalsize
		\label{fig:pinout}
	\end{figure}
	\end{appendices}

\end{document}