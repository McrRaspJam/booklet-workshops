\setcounter{section}{-1}
\section{Installing }

	%Workshop Description
	If you attended Workshop 12 on the Linux command line, and would like to have a go at a hands-on activity, this exercise will teach you to create your own personalised version of Raspbian.
	
	This exercise is designed to be completed mostly on your own, but do ask if you get stuck at any point!
	
	The contents of this booklet are an edited version of a guide written on the Raspberry Pi forums by Jimmy Ochoa, used with permission.
	The original post can be found at \label{source} \url{https://www.raspberrypi.org/forums/viewtopic.php?f=66&t=133691}.
	

	\subsection*{Raspbian Lite}

	If you visit the Raspberry Pi downloads sections, you'll find that there are two different versions of Raspbian available to download.
	
	Raspbian Lite is a version of the Raspbian operating system, with much of the typically included software left out.
	This includes some obvious candidates like Minecraft and Scratch, but even goes as far as to include no graphical desktop at all!
		
	\subsection*{How to use these Booklets}
	
	%Code Listings
	Terminal commands are listed as such. Copy everything \textit{after} the dollar sign. Lines without dollar signs are example outputs, and do not need to be copied.
			
\begin{lstlisting}
$ echo Hello, World!
Hello, World!
\end{lstlisting}
	
	All of our workshop resources are available to download from a Google Drive at
\url{http://bit.ly/mcrraspjam}.
There, you can find a PDF copy of this booklet, as well as template and completed program files for each workshop.
	
	%Aknowledgements
	These booklets were created using {\fontfamily{rfdefault}\selectfont \LaTeX}, an advanced typesetting system used for several sorts of books, academic reports and letters. Source files are available at \url{http://github.com/McrRaspJam/booklet-workshops}
		
	%License spiel
	%To allow modification and redistribution of these booklets, they are distributed under the \hbox{CC BY-SA 4.0} License. 	
	
	\subsection*{What you'll need}
		
		A Raspberry Pi, and a MicroSD card flashed with Raspbian Lite. If you're attending the February 2017 Jam, we'll have some of these for you to borrow.
		
	\subsection*{Questions?}
		If you get stuck, find errors or have feedback about these booklets, email:
		\url{jam@jackjkelly.com}\label{email}