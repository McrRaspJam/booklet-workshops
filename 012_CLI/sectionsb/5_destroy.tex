\section{Look South}

	The Linux command line places a lot of trust in you. As far as it's aware, you're somebody who knows what your doing. If you don't know what you're doing, be careful.
	
	We're going to have a look at some useful features of the command line, and see what happens when you combine them incorrectly.
		
	\subsection*{-f}
	
		\textbf{-f} stands for force, and on a command like rm skips any warning messages that would usually be created.
		
		if we were trying to delete 1000 files which were set as read-only, we would usually have to answer a yes/no check on every single file. -f will stop that from happening, so the command would finish instantly.
	
	\subsection*{sudo}
	
		\textbf{sudo} stands for superuser do. Any time we want to complete a command that requires a high level of authority, such as modifying the read-only files mentioned above, we run that command as a superuser using sudo.
		
		This command effectively means ``I know what I'm doing'', so make sure you do!
	
	\subsection*{Wildcards}
	
		In the CLI, the asterisk character \textbf{*} is used as a wildcard. If we wanted to delete all the .txt files from a directory filled with thousands of different files, we could find each one and manually type in
	
\begin{lstlisting}
$ rm file23of9000.txt
$ rm file72of9000.txt
$ rm file73of9000.txt
$ rm file79of9000.txt
...
\end{lstlisting}

		or, using a wildcard, we can search for a partial string. Entering:
\begin{lstlisting}
$ rm *.txt
\end{lstlisting}

		would delete all of the files ending in `.txt', just like we wanted
\begin{lstlisting}
$ rm january*
\end{lstlisting}

		would delete all file names beginning with `january'
	
		But what if you tried a command such as `rm *'? By default it would stop within the current folder, and possibly the sub folders, but what if we accidentally told it not to?
	
	\subsection{The ultimate command of destruction}
	
	The truly scary thing about the ultimate command of destruction is how inconspicuous it looks.
	
\begin{lstlisting}
$ sudo rm -rf /
\end{lstlisting}

	Let's break the command down.
	
	rm is our standard remove file function. To it, we have supplied the flags -rf. -f means we know what we're doing, and -r means it will start at one point and work through every subdirectory it finds.
	
	This is fine if we're trying to delete a Java Project folder, or a collection of cat pictures, but what directory have we supplied?
	
	/ is the root directory, the directory under which every single file on our computer is contained.
	
	The Operating system would have stopped you, but we've ran it under sudo, so the computer's convinced we done this intentionally.
	
	And now (if you're happy to wipe whatever SD card you're running on) you can try it on your Raspberry Pi. I'd recommend adding the flag -v for dramatic effect, which will display the files as they're being deleted.
	
\begin{lstlisting}
$ sudo rm -rfv /
\end{lstlisting}

\section{The End}

	Whilst it certainly is for your Raspberry Pi (until you reinstall NOOBS, that is), I thought I'd leave you with one story about the ultimate command of destruction.
	
	This story about a quite famous film company is available as a video at \url{https://youtu.be/8dhp_20j0Ys}, and a text article is available at \url{http://bit.ly/2jgR3nC}