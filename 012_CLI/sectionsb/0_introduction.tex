\setcounter{section}{-1}
\section{Introduction}
	
	%Workshop Description	
	If you were here for the first part of this workshop, you took a whistle-stop tour of everything on the PIXEL desktop. On a Linux computer however, the desktop only scratches the surface of the programs and functionality available.).
	
	The workshop introduces the basics of using the Linux Command Line Interface (or \textbf{CLI}), primarily to be able to configure and maintain our Pi's software.

	%Difficulty
	This is an intermediate Workshop. Once you're familiar with using your Pi on the desktop, you should be ready to attempt this workshop.
		
	\subsection*{About these booklets}
	
	%Code Listings
	Terminal commands are listed as such, copy everything \textit{after} the dollar sign. Lines without dollar signs are example outputs, and do not need to be copied.
			
\begin{lstlisting}
$ cd code
$ python3 helloworld.py 
Hello, World!
Hello, World!
\end{lstlisting}
	Occasionally, a concept will be explained in greater detail in \textit{asides}, like the one below. You can read these as you wish, but they're not required to complete the workshop.
	
\begin{aside}[Cryptography]
	We call the creation and study of ciphers \textit{cryptography}. `Crypto-' comes from the greek word \textit{kruptos}, meaning `hidden'.
\end{aside}
		
	%Aknowledgements
	These booklets were created using {\fontfamily{rfdefault}\selectfont \LaTeX}, an advanced typesetting system used for several sorts of books, academic reports and letters. The schematic diagrams were created using \textit{Fritzing}, a free hardware prototyping and PCB design software.
		
	%License spiel
	To allow modification and redistribution of these booklets, they are distributed under the \hbox{CC BY-SA 4.0} License. LaTeX source documents are available at \url{http://github.com/McrRaspJam/booklet-workshops}
	
	
	\subsection*{What you'll need}
		
		Just a working Raspberry Pi running Raspbian
		
	\subsection*{Questions?}
		If you get stuck, find errors or have feedback about these booklets, email:
		\url{jam@jackjkelly.com}\label{email}